\documentclass[man, fleqn, noextraspace]{apa6}
\usepackage{lmodern}
\usepackage{amssymb,amsmath}
\usepackage{ifxetex,ifluatex}
\usepackage{fixltx2e} % provides \textsubscript
\ifnum 0\ifxetex 1\fi\ifluatex 1\fi=0 % if pdftex
  \usepackage[T1]{fontenc}
  \usepackage[utf8]{inputenc}
\else % if luatex or xelatex
  \ifxetex
    \usepackage{mathspec}
  \else
    \usepackage{fontspec}
  \fi
  \defaultfontfeatures{Ligatures=TeX,Scale=MatchLowercase}
\fi
% use upquote if available, for straight quotes in verbatim environments
\IfFileExists{upquote.sty}{\usepackage{upquote}}{}
% use microtype if available
\IfFileExists{microtype.sty}{%
\usepackage{microtype}
\UseMicrotypeSet[protrusion]{basicmath} % disable protrusion for tt fonts
}{}
\usepackage{hyperref}
\hypersetup{unicode=true,
            pdftitle={Data Visdualization on Madarin Vowels},
            pdfauthor={Teresa Chen, Jun Lang, Steffi Hung, \& Ting-fen Lin},
            pdfkeywords={Madarin, Vowels, Native speaker, Non-native speaker},
            pdfborder={0 0 0},
            breaklinks=true}
\urlstyle{same}  % don't use monospace font for urls
\usepackage{graphicx,grffile}
\makeatletter
\def\maxwidth{\ifdim\Gin@nat@width>\linewidth\linewidth\else\Gin@nat@width\fi}
\def\maxheight{\ifdim\Gin@nat@height>\textheight\textheight\else\Gin@nat@height\fi}
\makeatother
% Scale images if necessary, so that they will not overflow the page
% margins by default, and it is still possible to overwrite the defaults
% using explicit options in \includegraphics[width, height, ...]{}
\setkeys{Gin}{width=\maxwidth,height=\maxheight,keepaspectratio}
\IfFileExists{parskip.sty}{%
\usepackage{parskip}
}{% else
\setlength{\parindent}{0pt}
\setlength{\parskip}{6pt plus 2pt minus 1pt}
}
\setlength{\emergencystretch}{3em}  % prevent overfull lines
\providecommand{\tightlist}{%
  \setlength{\itemsep}{0pt}\setlength{\parskip}{0pt}}
\setcounter{secnumdepth}{0}
% Redefines (sub)paragraphs to behave more like sections
\ifx\paragraph\undefined\else
\let\oldparagraph\paragraph
\renewcommand{\paragraph}[1]{\oldparagraph{#1}\mbox{}}
\fi
\ifx\subparagraph\undefined\else
\let\oldsubparagraph\subparagraph
\renewcommand{\subparagraph}[1]{\oldsubparagraph{#1}\mbox{}}
\fi

%%% Use protect on footnotes to avoid problems with footnotes in titles
\let\rmarkdownfootnote\footnote%
\def\footnote{\protect\rmarkdownfootnote}


  \title{Data Visdualization on Madarin Vowels}
    \author{Teresa Chen\textsuperscript{3}, Jun Lang\textsuperscript{2}, Steffi
Hung\textsuperscript{2}, \& Ting-fen Lin\textsuperscript{1}}
    \date{}
  
\shorttitle{Final Project in EDLD 610: Introduction to Data Science with R}
\affiliation{
\vspace{0.5cm}
\textsuperscript{1} Department of Human Physiology\\\textsuperscript{2} Department of East Asian Languages \& Linguistics\\\textsuperscript{3} Communication Disorders \& Sciences}
\keywords{Madarin, Vowels, Native speaker, Non-native speaker}
\usepackage{csquotes}
\usepackage{upgreek}
\captionsetup{font=singlespacing,justification=justified}

\usepackage{longtable}
\usepackage{lscape}
\usepackage{multirow}
\usepackage{tabularx}
\usepackage[flushleft]{threeparttable}
\usepackage{threeparttablex}

\newenvironment{lltable}{\begin{landscape}\begin{center}\begin{ThreePartTable}}{\end{ThreePartTable}\end{center}\end{landscape}}

\makeatletter
\newcommand\LastLTentrywidth{1em}
\newlength\longtablewidth
\setlength{\longtablewidth}{1in}
\newcommand{\getlongtablewidth}{\begingroup \ifcsname LT@\roman{LT@tables}\endcsname \global\longtablewidth=0pt \renewcommand{\LT@entry}[2]{\global\advance\longtablewidth by ##2\relax\gdef\LastLTentrywidth{##2}}\@nameuse{LT@\roman{LT@tables}} \fi \endgroup}


\DeclareDelayedFloatFlavor{ThreePartTable}{table}
\DeclareDelayedFloatFlavor{lltable}{table}
\DeclareDelayedFloatFlavor*{longtable}{table}
\makeatletter
\renewcommand{\efloat@iwrite}[1]{\immediate\expandafter\protected@write\csname efloat@post#1\endcsname{}}
\makeatother

\authornote{Steffi
and Jun are the owner of the dataset. They have the correct permissions
to make the dataset public.

Correspondence concerning this article should be addressed to Teresa
Chen, Rm.52 Gerlnger Annex, University of Oregon, OR 9740. E-mail:
\href{mailto:szuhuac@uoregon.edu}{\nolinkurl{szuhuac@uoregon.edu}}}

\abstract{
Several studies have examined consonant-vowel boundaries in sentences
and concluded that vowels contribute more than consonants to sentence
intelligibility. Since Mandarin has a greater proportion of vowels than
consonants, 35 vowels and 21 consonants, vowels indeed play an important
role for phonemic contrasts.This study examines six female native
speakers and six female non-native speakers' vowel production in
Mandarin. The native speakers' vowel space is compared with each other
and the Standard Chinese vowel chart to explore similar patterns. Also,
the non-native speakers' vowel space is compared with the native
speakers' vowel space and the Standard Chinese vowel chart to examine
native-like or non-native-like patterns. 
}

\begin{document}
\maketitle

{
\setcounter{tocdepth}{5}
\tableofcontents
}
\newpage

\section{Introduction}\label{introduction}

It is generally agreed that Mandarin Chinese has a five-vowel system
(see Hinton, Nichols, \& Ohala, 2006). These five vowels are {[}yi{]},
{[}yu{]}, {[}wu{]}, {[}en{]} and {[}ai{]}. Among these vowels, the mid
vowel has four allophones: {[}e{]}, {[}o{]}, {[}en{]} and {[}e{]}; and
the low vowel has two allophones: {[}ai{]} and {[}ao{]}. However, little
attention has been paid to the individual variances when producing these
nine vowels. Few researchers did empirical studies to examine Ashby and
Maidment (2005) vowel space of Chinese. In order to fill these gaps,
this study investigates the vowel distribution (figure 1) of native
Chinese speakers, aiming to determine whether native Chinese speakers
show similar patterns when producing Chinese vowels and whether their
patterns look similar to Roach's (2004) proposal of Chinese vowel chart.
In addition, this current work examines the vowel distribution of
American English learners of Chinese, with the purpose of finding out
whether non-native speakers perform similarly to native speakers in the
vowel production.

\begin{figure}
\centering
\includegraphics{picture/mouth.jpg}
\caption{Vocal Tract. Retrieved from
\url{http://www.uni-bielefeld.de/lili/personen/vgramley/teaching/HTHS/acoustic_2010.html}.
by Bielefeld University}
\end{figure}

\section{Methods}\label{methods}

\subsection{Participants}\label{participants}

Six female native (NS) Mandarin speakers and six female non-native (NNS)
Mandarin speakers participated in the study. The mean age of the NS
Mandarin speakers is 22.33 (range: 23-30) and that of the NNS Mandarin
speakers is 26.33 (range: 18-28). Among NS Mandarin speakers, two
speakers are from northern Mainland China (Beijing and Tianjin), three
speakers were from southern Mainland China (Nanjing, Chengdu and
Chongqing), and one speaker was from Taiwan. The Taiwanese participant
identified Mandarin as her most fluent language. All the six NNS
Mandarin speakers' native language was American English. They were all
novice-low learners who enrolled in first-year accelerated Chinese
language course at the same university. They had learned Mandarin for
six months and none of them had any study-abroad experience. See Figure
5-6 for participant demographics.

\subsection{Speech materials}\label{speech-materials}

We prepared nine Chinese sentences for speech materials. Each sentence
includes one of the following nine vowels: {[}yi{]}, {[}yu{]}, {[}wu{]},
{[}e{]}, {[}o{]}, {[}en{]}, {[}e{]}, {[}ai{]}, {[}ao{]}. Each vowel
appears after the aspirated bilabial stop {[}p{]} with a high tone (55).

\subsection{Procedure}\label{procedure}

Productions were elicited in a sentence-repetition oral task. Non-native
speakers (NNS) and native Chinese speakers (NS) were asked to read the
sentences twice. All participants read the speech materials for practice
once before recording. Recordings were made in a quiet study room in the
library using Praat Sound Recorder with 44,100 Hz sampling frequency,
and then these recordings were saved as wav files on a laptop. Formant 1
(F1) and Formant 2 (F2) were measured in the vowel mid-point for each
vowel shown in the spectrogram. All measurement was conducted using
Praat program on the same laptop. After the measurement, the mean F1 and
F2 values were plotted in charts using the program R to generate vowel
distribution for each speaker.

\section{Results and discussion}\label{results-and-discussion}

\subsection{Native speaker patterns}\label{native-speaker-patterns}

Table 1 shows the mean F1 and F2 values of nine Chinese vowels for
native speakers. Regarding F2 values, data shows that {[}yi{]} had the
highest F2 values for all native speakers, and {[}wu{]} had the lowest
F2 values for NS2 and NS3, but not for NS1. As for F1 values, three NS
also had different lowest and highest values. While {[}yi{]} had the
lowest F1 value and {[}ao{]} had the highest F1 value for NS1 and NS3,
NS2's {[}yu{]} had the lowest F1 value and her {[}ai{]} had the highest
F1 value.

\begin{table}

\caption{\label{tab:table1}Formant by volwels among non-native and native groups}
\centering
\begin{tabular}[t]{llrrrrrrrrr}
\toprule
Mean & group & ai & ao & e & en & wo & wu & ye & yi & yu\\
\midrule
F1 & NNS & 913.33 & 901.50 & 640.33 & 651.33 & 551.50 & 416.00 & 520.67 & 335.67 & 321.50\\
F1 & NS & 910.33 & 848.00 & 596.83 & 717.67 & 554.50 & 335.00 & 556.83 & 308.33 & 309.17\\
F2 & NNS & 1513.83 & 1377.00 & 1702.33 & 1980.00 & 1043.67 & 1122.83 & 2320.00 & 2646.50 & 1806.83\\
F2 & NS & 1655.50 & 1305.17 & 1289.67 & 1850.50 & 908.67 & 840.00 & 2486.50 & 2916.17 & 2420.83\\
\bottomrule
\end{tabular}
\end{table}

Clearer vowel distribution for each native speaker can be seen in the
formant plots (Figure 2-4). According to the \enquote{vowel dispersion
principle}, the vowel quadrilateral can be viewed as \enquote{a
perceptual space in which vowels are located in the oral cavity} (Ashby
and Maidment, 2005). In this study, the vowel quadrilateral is shown as
formant plots where the Y-axis is F1 (Hz) that corresponds to the height
of the tongue position; while the X-axis is F2 (Hz) that indicates the
backness of the tongue position for each vowel (Figure 2-4).

\begin{figure}
\centering
\includegraphics{Vowel_v2_files/figure-latex/figure1-1.pdf}
\caption{}
\end{figure}

\begin{figure}
\centering
\includegraphics{Vowel_v2_files/figure-latex/figure2-1.pdf}
\caption{}
\end{figure}

\begin{figure}
\centering
\includegraphics{Vowel_v2_files/figure-latex/figure3-1.pdf}
\caption{}
\end{figure}

\begin{figure}
\centering
\includegraphics{Vowel_v2_files/figure-latex/figure5-1.pdf}
\caption{}
\end{figure}

\begin{figure}
\centering
\includegraphics{Vowel_v2_files/figure-latex/figure6-1.pdf}
\caption{}
\end{figure}

\subsection{Second language
examination}\label{second-language-examination}

\section{Conclusion}\label{conclusion}

\newpage

\section{References}\label{references}

\begingroup
\setlength{\parindent}{-0.5in} \setlength{\leftskip}{0.5in}

\hypertarget{refs}{}
\hypertarget{ref-ashby2005}{}
Ashby, M., \& Maidment, J. (2005). \emph{Introducing phonetic science}.
Cambridge University Press.

\hypertarget{ref-hinton2006}{}
Hinton, L., Nichols, J., \& Ohala, J. J. (2006). \emph{Sound symbolism}.
Cambridge University Press.

\hypertarget{ref-roach2004}{}
Roach, P. (2004). British english: Received pronunciation. \emph{Journal
of the International Phonetic Association}, \emph{34}(2), 239--245.

\endgroup


\end{document}
